\documentclass[a4paper,10pt]{article}

% preamble
\usepackage[utf8]{inputenc} 
\usepackage[english]{babel} 
\usepackage{hyperref}

\title{Argumentation Mining}
\author{Filip Boltuzic}
\date{2018 \\ March}

\begin{document}

\maketitle

\tableofcontents

\section{Definition of Argumentation Mining}

Argument(ation) Mining is the automatic identification of the argumentative
structure contained withing in a piece of language~\cite{Lawrence2017}.

Argument mining is automatic extraction of arguments from natural text
\cite{Aker2017}.

\section{Tasks of Argumentation Mining}

\begin{enumerate}
  \item Identifying argumentative segments in text~\cite{Ajjour2017, stab2017argumentative}

  \item Clustering recurring arguments~\cite{boltuvzic2015identifying,
   misra2017using} 
   
  \item Recognizing argument schemes~\cite{feng2011classifying}. 

  \item Prediction of structure (connecting premises to claims)~\cite{Aker2017,
   Lawrence2017}

 \item Claim detection (can be similar to argument segments in text)~\cite{Levy2017}. 
   Can be context dependent~\cite{levy2014context} or independent~\cite{lippi2015context}
   
\end{enumerate}

\subsection{Identifying argumentative segments in text}
\label{subsec:argseg}

\noindent Unit segmentation consists in the splitting of a text into its
argumentative segments (ADU) and their non-argumentative counterparts 
\cite{Ajjour2017}.

\cite{persing2016end} rely on handcrafted features based on the parse tree
of a sentence to identify segments. 
\cite{stab2017argumentative} uses sequence modeling and sophisticated
features to classify the argumentativeness of each single word based on
its surrounding words. 
\cite{eger2017neural} employ a deep learning architecture using different
features based on the entire essay.
\cite{al2016news} have a rule-based where they suggest where the arguments
should be split before the actual argument annotation (annotators could
merge arguments back).
\cite{Aker2017} determine if a sentence is a claim,
premise or none. They work on a sentence boundary. 
\cite{Ferrara2017} model their approach as \emph{attraction to topics}. 
In an unsupervised setting, they claim an argumentative unit 
is something that is highly related to one topic only working on a 
sentence level. 

\subsection{Clustering recurring arguments}

\subsection{Prediction of structure}

\cite{Lawrence2017} have annotated debates on ``Moral Maze'' and created
argument diagrams via AIFDB\@. They aim to recognize the support relation
from text (inference or non-inference).
\cite{Aker2017} use claim-premise pairs and go full Cartesian on them,
making negative examples for those who aren't linked in the gold set. They
work on the~\cite{stab2017parsing, aharoni2014benchmark} datasets.
\cite{Hou2017} predict support/attack relations and stance classification at the same
time (jointly). They assume arguments that attack each other 
have opposing stances. 

\subsection{Claim detection}

\cite{Ferrara2017} use topic modeling to predict which sentence is a 
claim, major claim or premise. 

\section{Unsupervised approaches to Argumentation Mining}

Lack of large datasets for argumentation mining is one of the largest
concerns of the community. 

\cite{habernal2015exploiting} try to use unsupervised features for
better argument component identification from online debate portals. 
\cite{al2016cross} apply distant supervision to automatically create
a large annotated corpus from online debate portals with argumentative and
non-argumentative segments from several domains. 
\cite{Lawrence2017} try to use web search in combination with
\emph{therefore and because} discourse indicators in addition to some
other filtering. They make their own premise-conclusion pairs by searching
the web for the discourse marker and then use LDA to predict
support/non-support relations.
\cite{Levy2017} do unsupervised claim detection where they extract
sentences with ``that'' words in them and use 
them for claim detection. They acquire the sentences 
from Wikipedia (which is kind of distant supervision).
They evaluate their work through crowdsourced data labels. 
\cite{Ferrara2017} try to identify argumentative relations (see
Subsection~\ref{subsec:arg}) using unsupervised topic modeling. 

\section{Predicting support relations}

Predicting support relations is similar to textual entailment, but
involves more contextual knowledge and common-sense reasoning since the
semantic distance is greater. Also, it is not strictly a logical relation
and (with a well-defined hypothesis-text relation), but (usually) there
is a direction defined.
\cite{Lawrence2017} constructs a corpus using web-search and a gold set
then does supervised classification whether a sentence supports (infers)
another. 

\section{Ontology-based approaches to argumentation mining}

\cite{Szabo2018} has an ontology on climate change and wants to solve 
textual entailment on debate texts. 
They extract t-h pairs from debate sites (which have annotated pro/cons)
and use the ontology to improve classification (quite common knowledge
base aid approach). 

\section{Cross-domain argumentation mining}

\cite{Ajjour2017} do argumentative unit segmentation on three corpuses:
Habernal's Web Discourse, Stab's Essay corpus, and Editorials to show
how cross-domain argumentative unit segmentation is a huge problem as it
is defined today and even end with open questions about how should
segmentation and argumentative units be defined. 

\bibliographystyle{apalike}
\bibliography{argumentation_mining}

\end{document}
